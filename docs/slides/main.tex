\documentclass{beamer}

\mode<presentation> {
	\usetheme{Antibes}
}
\setbeamertemplate{footline}[page number]
\usepackage{nth}
\usepackage{graphicx}
\usepackage[font=small,labelfont=bf]{caption}
\usepackage{fancyvrb}
\graphicspath{ {images/} }
\usepackage{diagbox}
\usepackage{subfiles}
\usepackage{booktabs} % Allows the use of \toprule, \midrule and \bottomrule in tables

\title[Evolutionary Computing]{Relay Placement in Wireless Underground Sensor Networks} % The short title appears at the bottom of every slide, the full title is only on the title page

\author{Phan Ngoc Lan - 20142505\\ Nguyen Duy Manh - 20142857\\ Dinh Anh Dung - 20140774}
\institute[HUST]
{
Hanoi University of Science \& Technology  \\
School of Information \& Communications Technology
}
\date{\today}

\AtBeginSubsection[]
{
  \begin{frame}<beamer>{Overview}
    \tableofcontents[currentsection,currentsubsection]
  \end{frame}
}

\begin{document}

\begin{frame}
\titlepage
\end{frame}

\begin{frame}
\frametitle{Overview}
\tableofcontents
\end{frame}

\subfile{intro.tex}
\subfile{heuristics.tex}
\section{Proposed Methods}
\subfile{propose1.tex}
\subfile{propose2.tex}
\subfile{results.tex}

\begin{frame}{Conclusion}
	\begin{itemize}
		\item Heuristics provide very good running time but tend to be sub-optimal.
		\item GA approaches can create more optimal solutions in acceptable running time.
		\item {
			Future work:
			\begin{itemize}
				\item Report results in larger datasets.
				\item Explore other operators for 2 proposed algorithms.
				\item Explore methods for improving GA running time.
				\item Design new model 
    			\item Consider practical challenges (mobile relay nodes, multiple hops between RNs as well as SNs)
			\end{itemize}
		}
	\end{itemize}
\end{frame}

\end{document}
